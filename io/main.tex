\documentclass{article}
\usepackage [english, russian] { babel }

\begin{document}
\textbf{Неформальная постановка задачи}

Дано N независимых работ, для каждой работы задано время выполнения. Требуется построить расписание выполнения работ без прерываний на M процессорах. На расписании должно достигаться минимальное значение разбалансированности расписания (т.е. значения разности Tmax-Tmin, где Tmax - максимальное, по всем работам, время завершения работы в расписании; Tmin - аналогично, наименьшее время) (критерий К1).

\textbf{Формальная постановка задачи}

\textit{Дано}:
\begin{itemize}
\item Множество работ $P = \{p_1, p_2, ..., p_N\}$, где $p_i = \{N_i, W_i\}$, где $N_i$ -- номер работы, $W_i$ -- продолжительность работы.
\item Множество процессоров $M = \{m_i\}$.
\end{itemize}

Определим расписание $HP$ как пару $\{HP_B, HP_L\}$, где $HP_B: P \rightarrow M$ (каждой работе сопоставляется процессор, на котором она будет выполняться), а $HP_L = \{p_{i_j}\}$ -- упорядоченное множество, задающее порядок выполнения работ.

Введем обозначение множества $HP_L(k)$ как $[p_{i_0}, p_{i_1}, ..., p_k]$, то есть как множество работ, стоящих раньше $p_k$ в упорядоченном множестве $HP_L$, включая саму $p_k$ .

Определим множество $T$ как $\{\sum\limits_{p_i\in HP_L(k):HP_B(p_i) = HP_B(p_k)} W_i | p_k\in P\}$

\textit{Требуется}:
\begin{itemize}
\item Построить расписание $HP$.
\end{itemize}



\textit{Минимизируемый критерий}:
\begin{itemize}
\item $max(T) - min(T)$.
\end{itemize}

\textit{Ограничения}:
\begin{itemize}
\item $HP_B(p_k) = m_i,HP_B(p_k) = m_j;=> i = j.$
\item  Пусть $T_b(p_k)$- это время начала выполнения работы $p_k, T_e(p_k)$- время завершения работы $p_k$, тогда

$T_b(p_k)=\{\sum\limits_{p_i\in HP_L(k-1):HP_B(p_i) = HP_B(p_k)} W_i | p_k\in P\}$, а

$T_e(p_k)=\{\sum\limits_{p_i\in HP_L(k):HP_B(p_i) = HP_B(p_k)} W_i | p_k\in P\}$, то есть

$T_e(p_k)-T_b(p_k)=W_k$
\end{itemize}
\end{document}